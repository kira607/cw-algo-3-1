\section*{Обоснование выбора используемых структур данных}

Каждая нотация состоит из токенов типа \verb|Token|.
Конкретные типы токенов, будь то константа, операнд или функция,
наследуются от \verb|Token|.

Для строго ограниченных токенов
(операторы, функции, скобки, константы)
определены группы токенов и с их определениями.
Создание токенов из групп происходит
посредством вызова метода группы токенов.
Это необходимо для того, чтобы абстрагировать
создание токенов с особыми параметрами,
такими как приоритет операции,
ассоциативность (унарность),
коллбэк для операторов,
значение констант и т.д.

Определение (наследник класса \verb|Definition|)
содержит основные параметры для создания новых токенов из групп.

Для нотаций были реализованы три класса:

\verb|InfixExpression|,

\verb|PrefixExpression|,

\verb|PostfixExpression|.

Все классы для нотаций реализуют методы валидации и
вычисления конкретной нотации.
Общее для всех нотаций поведение содержится в базовом классе
\verb|BaseExpression|, включающее в себя
токенизацию строки с выражением с помощью класса \verb|Tokenizer|,
получение выражение в виде строки
и пр.

Для конвертации выражения одной нотации в другую
используется класс \verb|Converter|.

