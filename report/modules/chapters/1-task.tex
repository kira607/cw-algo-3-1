\section*{Задача}

Необходимо реализовать простейшую версию калькулятора.
Пользователю должен быть доступен ввод математического выражения,
состоящего из чисел и арифметических знаков.

Программа должна выполнить проверку корректности введенного выражения.
В случае некорректного ввода необходимо вывести сообщение об ошибке
с указанием позиции некорректного ввода.
В противном выводится обратная польская нотация введенного
выражения, а также отображается результат вычисления.

Входные данные:

\begin{itemize}
    \item арифметическое выражение
    \item поддерживаемый тип данных: вещественные числа (double)
    \item поддерживаемые знаки: $ +, -, *, /, ^, $ унарный $ - $, функции с одним
    аргументом ($ cos, sin, tg, ctg, ln, log, sqrt $ и др. (хотя бы одну не из списка)),
    \item константы pi, e открывающая и закрывающая скобки
\end{itemize}

Выходные данные:

\begin{itemize}
    \item постфиксная ФЗ
    \item результат вычисления
\end{itemize}

Входные данные по желанию можно читать из файла.